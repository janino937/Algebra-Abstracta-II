\documentclass[letter,twoside,12pt]{article}
\usepackage{lmodern}
\usepackage[T1]{fontenc}
\usepackage[spanish]{babel}
\usepackage[utf8]{inputenc}
\usepackage{amsmath}
\usepackage{amssymb}
\usepackage{amsthm}
\usepackage{fullpage}
\usepackage{latexsym}
\usepackage{enumerate}
\usepackage{enumitem}
\PassOptionsToPackage{hyphens}{url}\usepackage{hyperref}
\title{Algebra Abstracta: Tarea \#8}
\newtheorem{lemma}{Lema}
\author{Jonathan Andrés Niño Cortés}
\begin{document}
\maketitle
\textbf{Sección 13.1} \textbf{1.} Muestre que $ p(x) = x^3+9x+6 $ es irreducible en $ \mathbb{Q}[x] $. Sea $ \theta $ una raíz de $ p(x) $. Encuentre el inverso de $ 1 + \theta $ en $ \mathbb{Q}(\theta) $.

\begin{proof}
La irreducibilidad de este polinomio esta dada por el criterio de Einsenstein tomando el primo 3.

Ahora podemos escribir cualquier elemento de $ \mathbb{Q}(\theta) $ como $ a\theta^{2}+b\theta +c $. Entonces queremos encontrar los coeficientes $ a $, $ b $ y $ c $ tales que
\begin{equation}
(a\theta^{2}+b\theta +c)(1+\theta) = 1 \nonumber
\end{equation}

Desarrollando esta expresión obtenemos

\begin{equation}
a\theta^{2}+b\theta +c + a\theta^{3}+b\theta^{2} +c\theta = 1 \nonumber
\end{equation}

Pero como $ \theta $ es raíz de nuestro polinomio tenemos que $ \theta^3+9\theta+6= 0 $, es decir que $ \theta^{3}=-9\theta-6 $. Por lo tanto la expresión queda como

\begin{eqnarray}
a\theta^{2}+b\theta +c + a(-9\theta-6)+b\theta^{2} +c\theta ^2&=& 1 \nonumber
\\ (a+b)\theta^2 + (-9a+b+c)\theta +(-6a+c) &=& 1 \nonumber 
\end{eqnarray}

De aqui obtenemos el siguiente sistema de ecuaciones

\begin{eqnarray}
a+b &=&0 \nonumber
\\-9a+b+c&=&0 \nonumber
\\-6a+c&=&1 \nonumber
\end{eqnarray}

Resolviendo este sistema de ecuaciones obtenemos la solución $ a=1/4 $, $ b=-1/4 $ y $ c=5/2 $. Por lo tanto, $ 1/4\theta^2-1/4\theta+5/2 $ es el inverso de $ \theta+1 $.
\end{proof}

\textbf{Sección 13.1} \textbf{4.} Pruebe directamente  que el mapa $ a+b\sqrt{2} \mapsto a-b\sqrt{2} $ es un isomorfismo de $ \mathbb{Q}(\sqrt{2}) $ consigo mismo.
\begin{proof}
Sea $ \phi $ el mapa anterior que claramente esta bien definido. Probemos primero que preserva la suma

\begin{equation}
\phi(a+b\sqrt{2}+c+d\sqrt{2})=a+c-(b\sqrt{2}+d\sqrt{2})=a-b\sqrt{2}+c-d\sqrt{2}=\phi(a+b\sqrt{2})+\phi(c+d\sqrt{2})\nonumber
\end{equation}

Ahora probemos que preserva la multiplicación
\begin{equation}
\phi(a+b\sqrt{2})(c+d\sqrt{2}))=ac+2bd-(ad+bc)\sqrt{2}=(a-b\sqrt{2})(c-d\sqrt{(2)})=\phi(a+b\sqrt{2})\phi(c+d\sqrt{2}) \nonumber
\end{equation}

Ahora para probar que este homomorfismo solo basta probar que es diferente de 0. Y esto se puede ver porque $ \phi(1)=1 \not = 0 $.

Finalmente es sobreyectiva porque para cualquier $ a+b\sqrt{2} \in \mathbb{Q}(\sqrt{2}) $ tenemos que $ \phi(a-b\sqrt{2})=a+b\sqrt{2} $.
\end{proof}

\textbf{Sección 13.2} \textbf{3.} Determine el polinomio minimal sobre $ \mathbb{Q} $ del elemento $ 1+i $.

\begin{proof}
Si extendemos el campo a $ \mathbb{C} $ tenemos que $ x-1-i $ es un polinomio irreducible cuya raíz es la buscada. Si multiplicamos por el polinomio irreducible correspondiente al conjugado obtenemos

\begin{equation}
(x-1-i)(x-1+i)=x^2-x+ix-x+1-i-ix+i+1=x^-2x+2 \nonumber
\end{equation}
obtenemos un polinomio mónico cuyos coeficientes estan en $ \mathbb{Q} $ y que además es irreducible por el criterio de Einsenstein tomando $ p=2 $. Por lo tanto este polinomio es el polinomio minimal asociado a $ 1+i $. 
\end{proof} 


\textbf{Sección 13.2} \textbf{10.} Determine el grado de la extensión $ \mathbb{Q}(\sqrt{3+2\sqrt{2}}) $ sobre $ \mathbb{Q} $.

\begin{proof}
El punto anterior da la clave para resolver este ejercicio. Podemos demostrar que $ \sqrt{3+2\sqrt{2}} = 1+\sqrt{2}$.

En efecto, $ (1+\sqrt{2})^2=1+2\sqrt{2}+2 = 3+\sqrt{2} $, de donde se deduce la afirmación anterior al sacar raices a ambos lados.

Por lo tanto el polinomio mínimal asociado a $ 1+\sqrt{2} $ es

\begin{eqnarray}
x-1-\sqrt{2}&=&0 \nonumber
\\x-1&=&\sqrt{2} \nonumber
\\x^2-2x+1&=&2 \nonumber
\\x^2-2x-1&=&0 \nonumber
\end{eqnarray}

Este ultimo es el polinomio minimal pues es irreducible en $ \mathbb{Q} $. Para demostrar esto podemos extendernos al campo $\mathbb{R}$ donde este polinomio se descompone como $ (x-1-\sqrt{2})(x-1+\sqrt{2}) $. Esta descomposición es única porque $ \mathbb{R}[x] $ es un D.F.U. Por lo tanto, vemos que este polinomio no tiene raices en los racionales y como es de grado dos esto es lo único que basta para demostrar su irreducibilidad. Concluimos que la extensión es de grado 2.
 
\end{proof}

\textbf{Sección 13.2} \textbf{14.} Pruebe que si $ [F(\alpha):F] $ es impar entonces $ F(\alpha)=F(\alpha^2) $.

Para demostrar esto primero hacemos la siguiente observación. Si $ \beta \in F(\alpha) $ entonces $ F(\beta) \subseteq F(\alpha) $. Esto es por simple definición porque $ F(\beta) $ es la mínima extensión de campo que contiene a $ \beta $ y si ya esta contenida en la expansión de campo de $ \alpha $ entonces la extensión de $ \beta $ debe ser igual o menor a la de $ \alpha $.

Como claramente $ \alpha^2 \in F(\alpha) $ porque es la multiplicación de dos elementos en el campo concluimos que $ F(\alpha^2) \subseteq F(\alpha) $.

Para el otro lado vamos a demostrar que $ \alpha \in F(\alpha^2)$. Para esto necesitamos la suposición que $n =  [F(\alpha):F] $ es impar, es decir que existe $ k \in \mathbb{Z}_{>0} $ tal que $ n = 2k-1 $. Ahora esto es equivalente a que existe un polinomio en $ F[x] $ irreducible de grado $ n $ tal que $ \alpha $ es raíz.

Entonces tenemos la siguiente expresión $ (\alpha^2)^k = \alpha\alpha^n $.

Ahora sea $ x^n+a_{n-1}x^{n-1}+\cdots + a_0 $ el polinomio minimal de $ \alpha $. Como es irreducible tenemos que $ a_0 $ es diferente de 0 o de lo contrario sería divisible por $ x $. Como $ \alpha $ es raiz de aqui deducimos que $ \alpha^n =-(a_{n-1}\alpha^{n-1}+\cdots + a_1\alpha +a_0) $. Reemplazando esto en la primera expresión obtenemos que
$ (\alpha^2)^k = -\alpha(a_{n-1}\alpha^{n-1}+\cdots + a_0) = -(a_{n-1}\alpha^{n}+a_{n-2}\alpha^{n-1}\cdots + a_1\alpha^2+a_0\alpha) $.
Pero además podemos reescribir los $ n $ en términos de $ k $ para obtener la expresión $ (\alpha^2)^k =  -(a_{n-1}\alpha(\alpha^2)^{k-1}+a_{n-2}(\alpha^2)^{k-1}\cdots + a_1\alpha^2+a_0\alpha) $. Entonces podemos factorizar $ \alpha $ de los términos impares y obtener la expresión

$ \alpha(a_{n-1}(\alpha^2)^{k-1} + \cdots + a_0)  =  -( (\alpha^2)^k+a_{n-2}(\alpha^2)^{k-1}\cdots + a_1\alpha^2) $

Podemos pasar el término que multiplica a $ \alpha $ por cero porque es diferente de 0 pues $ a_0 \not = 0$. Por lo tanto
\begin{equation}
\alpha = -\frac{(\alpha^2)^k+a_{n-2}(\alpha^2)^{k-1}\cdots + a_1\alpha^2}{a_{n-1}(\alpha^2)^{k-1} + \cdots + a_0} \nonumber
\end{equation}

Por lo tanto logramos escribir $ \alpha $ como suma, multiplicación y división de elementos en $ F(\alpha^2) $. Concluimos que $ F(\alpha) \subseteq F(\alpha^2) $ y esto era lo que nos faltaba para concluir la igualdad. 

\textbf{Sección 13.2} \textbf{20.} Muestre que si la matriz de la transformación lineal "multiplicación por $ \alpha $" considerada en el ejercicio anterior es $ A $ entonces $ \alpha $ es una raíz del polinomio caracteristico para $ A $. Use este procedimiento para obtener el polinomio mónico de grado 3 satisfecho por $ \sqrt[3]{2} $ y por $ 1+\sqrt[3]{2}+\sqrt[3]{4} $.

\begin{proof}
Para probar esto vamos a utilizar el teorema de Calley-Hamilton. Este teorema indica que si $ P_f $ es el polinomio característico entonces $ P_{f}(f) = 0 $. En este caso $ f = \alpha Id $. Por lo tanto que $ P_f(f)=0 $ equivale a que $ P_f(\alpha)=0 $, es decir, que $ \alpha $ es una raíz del polinomio.

Para la segunda parte del problema necesitamos calcular las matrices asociadas a multiplicar por $ \sqrt[3]{2} $ y por $ 1+\sqrt[3]{2}+\sqrt[3]{4} $, respectivamente. Esto se hace calculando las transformaciones de los vectores canónicos $ 1, \alpha $ y $ \alpha^2 $. Para el primer caso la matriz va a ser

 \begin{equation}
 \begin{pmatrix}
 0 & 0 & 2
 \\1 & 0 & 0
 \\ 0 & 1 & 0
 \end{pmatrix} \nonumber
 \end{equation}

Y el polinomio característico para esta matriz es $ t^3-2 $ que es precisamente el polinomio minimal de $ \sqrt[3]{2} $.

Para el segundo caso la matriz asociada es 
 \begin{equation}
 \begin{pmatrix}
 1 & 2 & 2
 \\1 & 1 & 2
 \\ 1 & 1 & 1
 \end{pmatrix} \nonumber
 \end{equation}
 
 Y el polinomio caracteristico calculado es $ t^3-3t^2-3t-1 $ que tiene como raíz a $ 1+\sqrt[3]{2}+\sqrt[3]{4} $.
\end{proof}
\end{document}