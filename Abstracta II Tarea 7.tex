\documentclass[letter,twoside,12pt]{article}
\usepackage{lmodern}
\usepackage[T1]{fontenc}
\usepackage[spanish]{babel}
\usepackage[utf8]{inputenc}
\usepackage{amsmath}
\usepackage{amssymb}
\usepackage{amsthm}
\usepackage{amsthm}
\usepackage{fullpage}
\usepackage{latexsym}
\usepackage{enumerate}
\usepackage{enumitem}
\PassOptionsToPackage{hyphens}{url}\usepackage{hyperref}
\title{Topología: Tarea \#7}
\newtheorem{theo}{Teorema}
\newtheorem{lemma}[theo]{Lema}
\newtheorem*{defi}{Definición}
\author{Jonathan Andrés Niño Cortés}
\begin{document}
\maketitle
\textbf{Sección 10.2} \textbf{4.} Sea $ A $ cualquier $ \mathbb{Z} $-módulo, sea $ a $ cualquier elemento de $ A $ y sea $ n $ un entero positivo. Pruebe que el mapa $ \varphi_a: \mathbb{Z}/n\mathbb{Z} \to A $ dado por $ \varphi(\overline{k}) = ka $ es un homomorfismo de $ \mathbb{Z} $-módulos si y solo $ na = 0 $. Pruebe que $ \text{Hom}_\mathbb{Z}(\mathbb{Z}/n\mathbb{Z}, A) \cong A_n $, donde $ A_n = \{ a \in A \,|\, na = 0\}$ (de tal manera que $ A_n $ es el aniquilador de $ A $ del ideal $ (n) $ de $ \mathbb{Z} $).

\textbf{Sección 10.2} \textbf{8.} Sea $ \varphi: M \to N $ un homomorfismo de $ R $-módulos. Pruebe que $ \varphi(\text{Tor}(M)) \subseteq \text{Tor}(N)$.

\textbf{Sección 10.3} \textbf{10.} Asuma que $ R $ es conmutativo. Muestre que un $ R $-módulo es irreducible si y sólo si $ M $ es isomórfico (como un $ R $-módulo) a $ R/I $ donde $ I $ es un ideal máximal de $ R $. [Por el ejercicio previo, si $ M $ es irreducible entonces hay un mapa natural $ R \to M  $ definido por $ r \to rm  $, ,donde $ m $ es cualquier elemento no cero fijo de $ M $].

\textbf{Sección 10.3} \textbf{18.}
Sea $ R $ un dominio de ideal principal y sea $ M $ un $ R $-módulo que es aniquilado por el ideal propio no cero $ (a) $. Sea $ a = p_1^{\alpha_1}p_2^{\alpha_2}\cdots p_n^{\alpha_n} $ la factorización única de $ a $ en potencias de primos distintas en $ R $. Sea $ M_i $ el aniquilador de $ p_i^{\alpha_i} $ en $ M $, i.e., $ M_i $ es el conjunto $ \{ m \in M \,|\,p_i^{\alpha_i}m=0 \} $ --- llamado el \textit{componente $ p_i $-primario de $ M $}. Pruebe que
\begin{equation}
M = M_1 \oplus M_2 \oplus \cdots \oplus M_k. \nonumber
\end{equation}

\textbf{Sección 12.1} \textbf{13.} Si $ M $ es un módulo finitamente generado sobre el Dominio de Ideales Principales $ R $, describa la estructura de $ M/\text{Tor}(M) $.

\textbf{Sección 12.1} \textbf{13.} Sea $ R $ un D.I.P. y sea $ M $ un $ R $-módulo de torsión. Pruebe que $ M $ es irreducible si y sólo si $ M = Rm $ para cualquier elemento no cero $ m \in M $ donde el aniquilador de $ m $ es un ideal principal no cero $ (p) $. 


\end{document}