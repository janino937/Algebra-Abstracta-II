\documentclass[letter,twoside,12pt]{article}
\usepackage{lmodern}
\usepackage[T1]{fontenc}
\usepackage[spanish]{babel}
\usepackage[utf8]{inputenc}
\usepackage{amsmath}
\usepackage{amssymb}
\usepackage{amsthm}
\usepackage{fullpage}
\usepackage{latexsym}
\usepackage{enumerate}
\usepackage{enumitem}
\PassOptionsToPackage{hyphens}{url}\usepackage{hyperref}
\title{Algebra Abstracta: Tarea \#6}
\newtheorem{lemma}{Lema}
\author{Jonathan Andrés Niño Cortés}
\begin{document}
\maketitle
\textbf{Sección 9.5} \textbf{2.} Para cada uno de los cuerpos construidos en el Ejercicio 6 de la Sección 4 exhiba un generador para el grupo multiplicativo (cíclico) de elementos diferentes a cero.

\begin{proof}
Cada uno de estos grupos es cíclico por el teorema visto en clase.
\begin{enumerate}[label=\textbf{(\alph*)}]
\item 9). El anillo $ F_3[x]/(x^2+1) $ es un campo de orden 9, pues por un punto de la tarea anterior el polinomio $ x^2+1 $ es irreducible (ya que -1 no es residuo cuadrático módulo 3). Ahora vamos a demostrar que el polinomio $ x+1 $ es un generador de $ (F_3(x)/(x^2+1))^* $. Para ello vamosa demostrar que $ x+1 $ tiene orden 8.

\begin{eqnarray}
(x+1)^2 &=& x^2+2x+1 = 2x  \nonumber 
\\(x+1)^4 &=& 4x^2 = -1 \nonumber
\\(x+1)^8 &=& (-1)^2 = 1 \nonumber
\end{eqnarray}

\item 49). Tome el anillo $ F_7[x]/(x^2+1) $, que es un campo de orden 49. $ x^2+1 $ es de nuevo irreducible porque $ -1 $ no es residuo cuadrático módulo 7. El elemento $ x+2 $ es el generador del grupo de unidades, pues tiene orden 48.

\begin{eqnarray}
(x+2)^2 &=& x^2+4x+4 = 4x+3 \nonumber
\\(x+2)^3 &=& (4x+3)(x+2)=4x^2+11x+6=4x+2 \nonumber
\\(x+2)^4 &=& (4x+3)^2=16x^2+24x+9 = 3x \nonumber
\\(x+2)^6 &=& (4x+2)^2=16x^2+16x+4 = 2x+2 \nonumber
\\(x+2)^8 &=& (3x)^2=9x^2 = 5 \nonumber
\\(x+2)^{12} &=& 5*3x = x\nonumber
\\(x+2)^{16} &=& 5^2=25=4 \nonumber
\\(x+2)^{24} &=& x^2 = -1\nonumber
\\(x+2)^{48} &=& (-1)^2 = 1\nonumber
\end{eqnarray}
\item 8). Tome el anillo $ F_2[x]/(x^3+x+1) $ que es un campo porque el polinomio no tiene raices, pues evaluar tanto en 0 como en 1 da como resultado 1.

El grupo de unidades sería de orden 7. Como 7 es primo este grupo debe ser isomorfo a $ \mathbb{Z}/7\mathbb{Z} $. Por lo tanto cualquier polinomio diferente a 1 es un generador de este grupo.

\item 81). Tome el anillo $ F_3[x]/(x^4+x+2) $. $ x^4+x+2 $ es irreducible. En primer lugar no tiene raices pues si evaluamos en 0 da 2, si evaluamos en 1 da 1 y si evaluamos en 2 da 2. Queda la posibilidad que el polinomio sea divisible por polinomios irreducibles de grado 2. Pero es fácil ver que los únicos polinomios irreducibles de grado 2 son $ x^2+1 $, $ x^2+x+2$, $x^2+2x+2 $, $ 2x^2+x+1 $ $, 2x^2+2x+1 $ y $ 2x^2+2 $.

El elemento $ x+1 $ es el generador del grupo multiplicativo cíclico $ (F_3[x]/(x^4+x+2))^* $, para probar esto vamos a demostrar que $ x+1 $ tiene grado 80. Observese que tenemos la identidad $ x^4=-x-2 $.

\begin{eqnarray}
(x+1)^2&=&x^2+2x+1 \nonumber
\\(x+1)^4&=& x^4+4x^3+6x^2+4x+1 = x^3+2\nonumber
\\(x+1)^5&=& (x^3+2)(x+1) = x^3+x\nonumber
\\(x+1)^8&=& (x^3+2)^2 = x^6+4x^3+4=x^2+1\nonumber
\\(x+1)^{10}&=& (x^3+x)^2 = x^6+2x^4+x^2=2x^3+2x^2+x+2 \nonumber
\\(x+1)^{16}&=& (x^2+1)^2 = x^4+2x^2+1=2x^2+2x+2 \nonumber
\\(x+1)^{20}&=& (2x^3+2x^2+x+2)^2 =2x^3+2x^2+x \nonumber
\\(x+1)^{40}&=& (2x^3+2x^2+x)^2 = 2 \nonumber
\\(x+1)^{80}&=& (2)^2 = 4 = 1 \nonumber
\end{eqnarray} 
\end{enumerate}

\end{proof}
\textbf{Sección 9.5} \textbf{3.} Sea $ p $ un primo impar en $ \mathbb{Z} $ y sea $ n $ un entero positivo. Pruebe que $ x^n-p $ es irreducible sobre $ \mathbb{Z}[i] $.

\begin{proof}
Tenemos dos casos. Si $ p \equiv 3 $ mod 4, entonces $ p $ es irreducible. Por lo tanto si tomamos el ideal generado por $ p $, este ideal es primo y por el criterio de Eisenstein concluimos que $ x^n-p $ es irreducible. Ahora si  $ p \equiv 1 $ mod 4 entonces existen $ a,b \in \mathbb{Z} $ tales que $ p = (a+bi)(a-bi) $ y estos factores son irreducibles. Si tomamos el ideal primo $ (a+bi) $ vemos que $ p $ pertenece a $ (a+bi) $ pero no pertenece a $ (a^2+2abi-b^2) $ porque esto contradeciria la factorización única de $ p $. Luego por el criterio de Eisenstein concluimos que $ x^n-p $ es irreducible.
\end{proof}

\textbf{Sección 9.5} \textbf{4.}  Pruebe que $ x^3 +12x^2+18x+6$ es irreducible sobre $ \mathbb{Z}[i] $.

\begin{proof}
3 es irreducible en $ \mathbb{Z}[i] $ y por lo tanto su ideal es primo. Vemos que los coeficientes 12, 18 y 6 pertenecen a (3). Pero además, 6 no pertenece a (9) porque 6 tiene una norma más pequeña que 9. Luego por el criterio de Eisenstein concluimos que $ x^3+12x^2+18x+6 $ es irreducible. 
\end{proof}
\textbf{Sección 9.5} \textbf{7.} Pruebe que los grupos aditivos y multiplicativos de un campo nunca son isomórficos.

\begin{proof}
Considere 3 casos. Cuando el campo es finito claramente no son isomorfos porque si el orden del campo en $ n $, el orden del grupo aditivo es $ n $ mientras que el orden del grupo multiplicativo es $ n-1 $. 

Ahora, si el campo es infinito y $ -1 \not = 1 $, entonces tenemos que -1 es de orden 2 en el grupo multiplicativo, pero en el grupo aditivo no hay ningun elemento de orden 2. Si lo hubiera este deberia ser tal que $ x+x = 0 $, ahora si multiplicamos por $ x^{-1} $ a lado y lado obtenemos que 1+1=0. Luego $ 1 = -1 $ lo que contradice nuestra hipótesis.

Por otro lado, si $ -1 = 1 $ entonces eso quiere decir que 1 es su propio inverso aditivo.
Si tomamos cualquier homomorfismo $ \phi $ entre el grupo aditivo y el multiplicativo entonces $ \phi(0)=1 $, pero además $ \phi(1+1)=\phi(0)=1 $, por lo tanto $ \phi(1)\phi(1)=1 $. Ahora las únicas soluciones de la ecuación $ x^2 = 1 $ son 1 o -1 pero como $ -1 = 1 $ concluimos que $ \phi(1)=1 $ por lo que ningun homomorfismo puede ser inyectivo.
\end{proof}

\textbf{Sección 10.1} \textbf{8.} Un elemento $ m $ de un $ R$-módulo $ M $ se llama un elemento de torsión si $ rm=0 $ para algún elemento no cero $ r \in R $. El conjunto de los elementos de torsión se denota

\begin{equation}
\text{Tor}(M)=\{m \in M \,|\,rm = 0 \text{ para algún elemento no cero } r \in R \}.\nonumber
\end{equation}

\begin{enumerate}[label=\textbf{(\alph*)}]
\item Pruebe que si $ R $ es un dominio integral entonces Tor($ M $) es un submódulo de $ M $ (llamado el submódulo de torsión de $ M $).
\begin{proof}
Vamos a utilizar el criterio de submódulo. En primer lugar Tor($ M $) no es vacío pues para cualquier $ r \in R $ tenemos que $ r*0=0 $, por lo que 0 es un elemento de torsión.

En segundo lugar tomemos $ x+ry $ para cualquier $ r \in R $ y cualesquiera $ x,y \in N $ y veamos que pertenece a $ N $. Por definición tenemos que existen $ s,t \in R $ tales que $ sx=0 $ y $ ty=0 $, luego si multiplicamos por $ st $ tenemos que $ (st)(x+ry)=stx+stry=t(sx)+sr(ty)=t0+sr0=0 $. Por lo que $ x+ry \in N $.
\end{proof}
\item De un ejemplo de un anillo $ R $ y un $ R $-módulo $ M $ tal que Tor($ M $) no es un submódulo.
\begin{proof}
Tome por ejemplo el anillo que no es un dominio integral $ \mathbb{Z}/6\mathbb{Z} $, si tomamos a $ R $ como un $ R $-m'odulo entonces Tor$(M)$ sería igual a los divisores de 0 junto con el 0 de $ R $. Estos son $ \overline{0} $, $ \overline{2} $ y $ \overline{3} $. Pero esto no es un submódulo de $ R $, pues si tomamos la suma $ \overline{2} + \overline{3} = \overline{5}$, esta no pertenece a los elementos de torsión. 
\end{proof}
\item Si $ R $ tiene divisores de cero muestre que cualquier $ R $-módulo no cero tiene elementos de torsión no cero.
\begin{proof}
Sea $ M $ un $ R $-módulo. Tome $ a,b \in R$ diferentes de 0 tales que $ ab=0 $. Ahora tome cualquier elemento $ m \in M $ distinto de 0. Si $ bm = 0 $ ya tendriamos que $ m $ es un elemento de torsión, si no entonces $ a(bm)=abm = 0m = 0 $, por lo que $ bm $ sería un elemento de torsión.
\end{proof}
\end{enumerate}

\textbf{Sección 10.1} \textbf{18.} Sea $ F = \mathbb{R} $, sea $ V = \mathbb{R}^2 $ y sea $ T $ la transformación lineal de $ V $ a  $ V $ que es rotación horaria alrededor del origen por $ \pi/2 $ radianes. Muestre que $ V $ y 0 son los únicos $ F[x] $-submódulos para este $ T $.

\begin{proof}
Una forma de demostrarlo es calculando el polinomio caracteristico de la transformación. La matriz asociada a esta transformación es 
\begin{equation}
\begin{pmatrix}
\cos(\pi/2)  &  -\sin(\pi/2) \\
\sin(\pi/2)  &  cos(\pi/2) 
\end{pmatrix} =
\begin{pmatrix}
0  &  -1 \\
1  &  0 
\end{pmatrix}
\nonumber
\end{equation}

El polinomio caracteristico lo calculamos como

\begin{equation}
\begin{vmatrix}
t & -1 \\
1 & t
\end{vmatrix} = t^2+1 \nonumber
\end{equation}

Pero vemos que este polinomio es irreducible en $ \mathbb{R} $ por lo que esta transformación no tiene asociado ningun valor propio. Esto significa que no hay espacios invariantes de dimensión 1 por lo que los únicos $ F[x] $-módulos posibles son $ V $ y 0.
\end{proof} 
\end{document}