\documentclass[letter,twoside,12pt]{article}
\usepackage{lmodern}
\usepackage[T1]{fontenc}
\usepackage[spanish]{babel}
\usepackage[utf8]{inputenc}
\usepackage{amsmath}
\usepackage{amssymb}
\usepackage{amsthm}
\usepackage{fullpage}
\usepackage{latexsym}
\usepackage{enumerate}
\usepackage{enumitem}
\PassOptionsToPackage{hyphens}{url}\usepackage{hyperref}
\title{Algebra Abstracta: Tarea \#5}
\newtheorem{lemma}{Lema}
\author{Jonathan Andrés Niño Cortés}
\begin{document}
\maketitle

\begin{enumerate}
\item Sea $ F $ un campo y sea $ f(x) $ un polinomio no constante en $ F[x] $. Describa el nilradical de $ F[x]/(f(x)) $ en términos de la factorización de $ f(x) $

\begin{proof}
Como $ F $ es un campo tenemos que $ F[x] $ es un dominio euclideano y por lo tanto es un D.I.P y un D.F.U. En primer lugar existe una factorización única de $ f(x) $ en irreducibles. Sea dicha factorización $ f(x)=\pi_1^{\alpha_1} \cdots \pi_n^{\alpha_n} $. En un punto en una tarea anterior se demostró que el nilradical de $R/I$ es igual al $ \text{rad } I /I$ donde rad $ I $ es $ \{r \in R\;|\; r^n \in I\ \text{para algún } R \in \mathbb{Z}^+\} $

Para nuestro caso rad $(f(x)) = (\pi_1\cdots \pi_n)$. En efecto, para cualquier $ r \in  $ rad $ (f(x)) $ tenemos que $ r^n = q\pi_1^{\alpha_1} \cdots \pi_n^{\alpha_n}$. Entonces $r$ también es divisible por los mismos irreducibles que $r^n$ por lo que $ r \in (\pi_1 \cdots \pi_n) $ y además $ \pi_1 \cdots \pi_n \in  $ rad $ (f(x)) $ pues  $ \pi_1^{\alpha} \cdots \pi_n ^\alpha \in (f(x))$ para $ \alpha = \max(\alpha_i)$. 

Por lo tanto el nilradical de $ F[x]/(f(x)) $ es $ (\pi_1\cdots \pi_n)/(f(x)) $.
\end{proof}

\item \begin{enumerate}
\item Sean $ F \subseteq K $ dos cuerpos y sean $ p(x) $, $ q(x) $ dos polinomios diferentes de cero en $ F[x] $. Muestre que
\begin{equation}
\text{m.c.d}_{F [x]}(p, q) = 1 \Longleftrightarrow \text{m.c.d}_{K[x]}(p, q) = 1. \nonumber
\end{equation}
\begin{proof}

$ F[x] $ y $ K[x] $ son ambos dominios euclideos, y por lo tanto también son dominios de factorización única.

$ \Rightarrow $: $ \text{m.c.d}_{F [x]}(p, q) = 1 $ implica que existen $ r, s  \in P[x]$ tales que $ pr+qs=1$. Pero $ p $, $ r $, $ q $ y $ s $ pertenecen a $ K[x] $. Por lo tanto, $ \text{m.c.d}_{K [x]}(p, q) = 1 $.

Observación: Tanto en $ F[x] $ como en $ K[x] $ las unidades son los polinomios de grado 0, es decir las constantes. Esto es una consecuencia de la norma euclidea asociada a estos anillos.

$ \Leftarrow $: Si suponemos por contradicción que $ m= \text{m.c.d}_{F [x]}(p, q) \not = 1$ entonces $m$ es un polinomio de grado mayor a 0 que divide tanto a $p$ como a $q$. Entonces extendiendo a $ K[x] $, $ m $ sigue dividiendo tanto a $ p $ como a $ q $ y por lo tanto debe dividir a $ \text{m.c.d}_{K [x]}(p, q) $, que por lo tanto no puede ser igual a 1 porque $ m $ no es una unidad en $ K[x] $. 
\end{proof}
\item Sea $ R $ un anillo conmutativo con identidad. La derivada formal en el anillo de polinomios $ R[x] $ se
define de la manera usual i.e., si $ p(x) = a_0 + a_1x + \cdots + a_nx ^ n $ entonces
\begin{equation}
 p'(x) := a_1 + 2a_2x^+3a_3x^2 \cdots + a_nnx ^ {n-1} \nonumber
\end{equation}

Sea $ F $ un cuerpo contenido en los números complejos y $ p(x) \in F[x] $. Una raiz $ \alpha \in \mathbb{C} $ de $ p(x) $ se llama
\textit{raíz repetida} si $ (x - \alpha)^2 $  divide a $ p(x) $ en $ \mathbb{C}[x] $.

Sea $ p(x) \in F[x]\backslash{0} $ Muestre que m.c.d$ (p, p'
) = 1 $ si y sólo si $ p $ no tiene raíces repetidas.
\begin{proof}
Por el literal anterior, esto es equivalente a demostrar  que $ \text{m.c.d}_{\mathbb{C}[x]}(p, p') = 1 $ si y sólo si $ p $ no tienes raices repetidas.

$ \Rightarrow $: Suponga que $ p $ tiene raices repetidas. Entonces $ p(x) = (x-\alpha)^2q(x)$, donde $ \alpha \in \mathbb{C} $ y $ q(x) \in \mathbb{C} $. Por cálculo de variable compleja sabemos que la derivada es igual a $ p'(x) = 2(x-\alpha)q(x)+(x-\alpha)^2q'(x)$. Concluimos que $ (x-\alpha) $ divide tanto a $ p $ como a $ p' $. Luego $ \text{m.c.d}_{\mathbb{C}[x]}(p, p') \not = 1 $.

$ \Leftarrow $: Suponga que $ p(x) $ no tiene raices repetidas. Luego $ p = (x-\alpha_1)(x-\alpha_2)\cdots (x-\alpha_n) $. Además, sabemos que en $ \mathbb{C} $ los polinomios de grado 1 son irreducibles por lo que esta es la factorización única en irreducibles de $ p $. Entonces vamos a demostrar que níngun $ x- \alpha_i $ divide a $ p'(x) $ lo que implica que $ \text{m.c.d}_{\mathbb{C}[x]}(p, p') = 1 $, pues no tendrían factores irreducibles en común.

Obsérvese que $p(x)=(x-\alpha_i)q(x)$ donde $ q(x) $ no es divisible por $ (x-\alpha_i) $.
Si $ p'(x) $ fuera divisible por $ (x-\alpha_i) $ entonces $ q(x) = p'(x)-(x-\alpha_i)q'(x) $ sería divisible por $ (x-\alpha_i) $ lo cual es una contradicción. Por lo tanto $ p'(x) $ no es divisible por $ x- \alpha_i $. Esto concluye la demostración.
\end{proof}
\end{enumerate}
\item Sea $R$ un D.F.U. y sea $ a \in R $. De condiciones necesarias y suficientes sobre a de tal forma que $x^2-a$ sea
un polinomio irreducible en $ R[x] $.
\begin{proof}
La condición necesaria y suficiente para que $ x^2-a $ sea un polinomio irreducible en $ R[x] $ es que $ a $ no sea un cuadrado perfecto. Por un lado es necesaria porque si $ a=\alpha^2 $ entonces $ x^2-a =(x-\alpha)(x+\alpha)$. Por otro lado es suficiente pues si suponemos que $ x^2-a $ es reducible, entonces existen polinomios que deben ser de grado 1, $ (x+\alpha)(x+\beta) $ con $ \alpha,\beta \in R $ tales que $ (x+\beta)(x+\gamma) = x^2-1 $. Por lo tanto, $ x^2+(\alpha+\beta)x+\alpha\beta =x^2-1$ de donde deducimos que $ \alpha+\beta = 0  $ y $ \alpha\beta = -a$. Por lo tanto, $ \alpha = - \beta $ y $-a = \alpha(-\alpha)$ de donde concluimos que $ a = \alpha^2 $ es un cuadrado perfecto.

\end{proof}
\item Sea $ K $ un cuerpo y sea $ f(x) \in K[x] $ un polinomio de grado al menos 1 que no es un cuadrado perfecto.
Muestre que
\begin{equation}
K[x,y]/\langle y^2-f(x)\rangle\nonumber
\end{equation}
es un dominio.

\begin{proof}
Esto es una consecuencia del punto anterior, pues $ K[x,y]=K[x][y] $ y nuestra suposición nos dice que $ f(x) $ no es un cuadrado perfecto en $ K[x] $. Por lo tanto, $ y^2 -f(x)$ es irreducible en $ K[x,y] $. Pero además, como $ K $ es un cuerpo $ K[x] $ sería un dominio euclideo y por lo tanto un D.F.U. y $ K[x,y] $ sería un D.F.U. también por un teorema visto en clase. Por lo tanto, $ y^2 -f(x)$ también es primo y entonces $ K[x,y]/\langle y^2-f(x)\rangle\nonumber $ es un dominio.
\end{proof}

\item Sea $ R $ un dominio con cuerpo de fracciones $ K $. El anillo $ R $ se llama integralmente cerrado si todo $ r \in K
$ que es raiz de un polinomio mónico con coeficientes en $ R $ está en $ R $.

\begin{enumerate}
\item  Muestre que todo D.F.U es integrálmente cerrado
\begin{proof}
Tome $  p(x)  $ un polinomio mónico en $ R[x] $ y supongamos que $ r \in K $ es una raíz de $ p $. Entonces, $ p(x) $ como polinomio en $ K[x] $ es divisible por $ (x-r) $, es decir que existen $ q \in K[x] $ tal que $ p(x)=(x-r)q(x) $. Como $ p $ es mónico $ q $ también tiene que ser mónico. Luego si $ q $ fuera de grado 0 tendría que ser 1, y esto implicaria que $ p(x)=x-r $, pero como $ p(x) $ pertenece a $ R[x] $ esto implica que $ r \in R $. Si $ q(x) $ es de grado mayor entonces por el lema de Gauss (que podemos utilizar porque $ R $ es un D.F.U.) existen elementos $ r,s \in F $ tales que $ rq(x)$ y $ s(x-r) \in R[x] $ y $ p(x)=rq(x)s(x-r) $ pero como $ p(x) $ es mónico esto implica que $ rs =1 $, es decir $ p(x)=q(x)(x-r) $ con $ q(x) $ y $ (x-r) $ en $ R[x] $. Esto implica que $ r \in R $.
\end{proof}
\item  Muestre que el anillo $ \mathbb{C}[x,y]/\langle y^2-x^3\rangle $ es un dominio que no es integralmente cerrado.

\begin{proof}
Sea $ R = \mathbb{C}[x]\mathbb{C}[x,y]/\langle y^2-x^3\rangle $

$ x^3 $ no es cuadrado perfecto en $ \mathbb{C}[x] $. Por lo tanto, el punto 4 nos permite concluir que $ R $ es un dominio. Notese que $ x^3 \equiv y^2 $ mod $ \langle y^2-x^3\rangle$. Ahora para demostrar que no es integralmente cerrado tomese el polinomio mónico $ z^2-x $. Este polinomio es irreducible en $ R[z] $ pues claramente $x$ no es un cuadrado perfecto de $ R $. Pero si tomamos $ K[z] $ donde $ K $ es el campo de fracciones de $ R $. Entonces $ y/x  \in K$ sería una raiz del polinomio. En efecto $ (y/x)^2 -x = y^2/x^2-x$, pero $ y^2 = x^3 $, por lo tanto, $ y^2/x^2-x = x^3/x^2-x=x-x=0 $. Esto demuestra que $ R $ no es integralmente cerrado.
\end{proof}

\item  Sean $ m $ y $ n $ enteros positivos y suponga que $ m $ no es una potencia $ n $-ésima perfecta (por ejemplo 32
no es un cuadrado perfecto y 25 no es un cubo perfecto.) Muestre que $ \sqrt[n]{m} $ es irracional.

\begin{proof}
Por el punto a) tenemos que $ \mathbb{Z}[x] $ es integralmente cerrado. Entonces considere el polinomio $ x^n-m $, que es un polinomio mónico con coeficientes en $ \mathbb{Z} $. $ \sqrt[n]{m} $ es precisamente una raiz de este polinomio en $ \mathbb{R}[x] $. Ahora si suponemos que $ \sqrt[n]{m} $ es racional entonces por el literal a) tendriamos que $ \sqrt[n]{m} $ sería un entero. Es decir que $ m = z^n $ para $ z = \sqrt[n]{m} \in \mathbb{Z} $, lo que contradice nuestra suposición que $ m $ no es potencia $ n $-ésima perfecta.
\end{proof}
\end{enumerate}
\item Un anillo conmutativo con unidad $ R $ se llama anillo local si $ R $ tiene un único ideal maximal.

\begin{enumerate}
\item  Sean $ R $ un anillo conmutativo con identidad, $ M $ un ideal maximal de $ R $ y $ n $ un entero positivo.
Muestre que $ R/M^n $ es un anillo local.

\begin{proof}
Por un punto realizado en una tarea anterior tenemos que si $ N $ es un ideal máximal en $ R/M^n $ entonces $N^*=\pi^{-1}(N)$ es un ideal máximal en $ R $ donde $ \pi $ es el homomorfismo natural entre $ R  $ y $ R/M^n $ que claramente es sobreyectivo.

Pero vemos que $M^n \subseteq N^* $ y que además como $ N^* $ es máximal entonces es primo. Ahora por un teorema visto en clase concluimos que $ M  \subseteq N^* $ o que $ M^{n-1} \subseteq N^*$. Repitiendo este proceso $ n $ veces llegamos a que $ M  \subseteq N^* $. Pero como $M$ es máximal esto significa que $ N^* $ debe ser igual a $ M $. Por lo tanto, $ M/M^n $ es el único ideal máximal de $ R/M^n $, es decir que es un anillo local.
\end{proof}

\item  Sea $ R $ un D.I.P y sea $ I \not = 0 $ un ideal de $ R $. Muestre que $ R/I $ es isomorfo a un producto finito de
anillos locales.


\begin{proof}
En primer lugar, como $ R $ es un D.I.P también es un D.F.U. Si $I=(1)$ no hay nada que demostrar pues $ R/I = \{0\}$ es trivialmente un anillo local. Entonces, sea $ I = (\alpha) $ y sea $ \pi_1^{\alpha_1} \cdots \pi_n^{\alpha_n} $ la factorización de $ \alpha $ en irreducibles. Como $ \pi_i $ es irreducible entonces es primo, pero además como estamos en un D.I.P primo implica maximal. Luego cada $ (\pi_i) $ es máximal. Además $ (\pi_i^{\alpha_i}) $ y $ (\pi_j^{\alpha_j}) $ son comaximales pues sus generadores no tienen factores irreducibles en común. Luego por el teorema chino del residuo tenemos que $ R/I \cong R/(\pi_1^{\alpha_1}) \times \cdots \times R/(\pi_n^{\alpha_n})$ donde cada $ R/(\pi_i^{\alpha_i}) $ es un anillo local por el punto a).
\end{proof}
\end{enumerate}
\end{enumerate}
\end{document}