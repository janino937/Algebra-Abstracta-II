\documentclass[letter,twoside,12pt]{article}
\usepackage{lmodern}
\usepackage[T1]{fontenc}
\usepackage[spanish]{babel}
\usepackage[utf8]{inputenc}
\usepackage{amsmath}
\usepackage{amssymb}
\usepackage{amsthm}
\usepackage{fullpage}
\usepackage{latexsym}
\usepackage{enumerate}
\usepackage{enumitem}
\PassOptionsToPackage{hyphens}{url}\usepackage{hyperref}
\title{Algebra Abstracta: Tarea \#10}
\newtheorem{lemma}{Lema}
\author{Jonathan Andrés Niño Cortés}
\begin{document}
\maketitle
\begin{enumerate}
\item Sea $ L/K $ una extensión de cuerpos.
\begin{enumerate}

\item Sea $ p $ un primo y suponga que $ [L : K] = p $. Muestre que $ L/K $ es una extensión simple.

\begin{proof}
Tome cualquier elemento $ \alpha \in L $ tal que $ \alpha \not \in K $ y considere la extensión $ K(\alpha) $. Por el lema de Torres tenemos que $ p = [L:K]=[L:K(\alpha)][K(\alpha):K] $. Así que $ [K(\alpha):K] $ es un divisor de $ p $ y como por nuestra suposición no es 1, concluimos que $ [K(\alpha):K] = p $ y que $ [L:K(\alpha)]=1 $, es decir que $ L = K(\alpha) $ lo que muestra que $ L $ es simple.   
\end{proof}

\item Muestre que no existe $ L/\mathbb{C} $ tal que $ [L : \mathbb{C}]=2 $.

\begin{proof}
Suponga por contradicción que existe $ L/\mathcal{C} $ tal que $ [L : \mathbb{C}]= 2 $. Entonces debe existir algún elemento $ \alpha $ tal que su polinomio minimal sea de grado 2. Pero se puede demostrar (sin necesidad del resultado más fuerte que $ \mathbb{C} $ es algebraicamente cerrado) que todo polinomio de grado 2 en $ \mathbb{C} $ es reducible.

Esto es gracias a la ecuación cuadrática. Cualquier polinomio mónico de grado 2 $  x^2+ax+b $ en $ \mathbb{C} $ se puede factorizar en dos polinomios mónicos de grado 1 como $ (x-x_1)(x-x_2) $ donde

 \begin{equation}
 x_{1,2} = \frac{-a\pm\sqrt{a^2-4b}}{2} \nonumber
 \end{equation}
 
 $ x_{1,2}\in \mathbb{C} $ porque en los complejos la función raiz se puede defenir para cualquier número complejo a diferencia de los reales donde las raíces de números negativos no estan definidas.
 
 Llegamos a una contradicción y concluimos que no puede existir dicha extensión.
\end{proof}

\item Suponga que toda extensión finita de $ \mathbb{R} $ es simple. Muestre que no existe $ L $ extensión finita $ \mathbb{R} $ de
grado $ [L : \mathbb{R}] $ impar.

\begin{proof}
De nuevo suponiendo por contradicción que $ L $ es una extensión finita de $ \mathbb{R} $ de grado impar tendriamos que existe algún elemento $ \alpha \in L $ tal que su polinomio mínimal es de grado impar.

Pero en $ \mathbb{R} $ todo polinomio mónico impar tiene por lo menos una raiz lo que implica que es reducible. Para demostrar esto podemos utilizar el teorema de valor intermedio de cálculo. Sea $ P(x) $ el polinomio irreducible. Como el polinomio es mónico de grado impar tenemos que su limite al infinito es igual a infinito y su limite a menos infinito es menos infinito. Entonces podemos tomar un valor $ a $ tal que $ P(a)< 0 $ y un valor $ b $ tal que $ P(b) > 0 $. Entonces por el teorema del valor intermedio existe un valor $ c $ entre $ a $ y $ b $ tal que $ P(c)=0 $, es decir, que $ c $ es una raíz de $ P(x) $.
\end{proof}
\end{enumerate}

\item Sea $ L/K $ una extensión de cuerpos y sea $ \alpha \in L $.

\begin{enumerate}
\item Muestre que $ \alpha $ es algebraico sobre $ K $ si y sólo si $ K[\alpha]  = K(\alpha ) $. (Acá $ K[\alpha] $ denota el sub-anillo de $ L $ generado por $ K $ y $ \alpha $.)
\begin{proof}
Una desigualdad siempre se cumple. Tenemos que $ K[\alpha] \subseteq K(\alpha)  $, pues cualquier elemento en $ K[\alpha] $ se puede escribir como $ a_n\alpha^n+\cdots+a_1\alpha+a_0 $ donde $ a_i \in K $ y claramente esto pertenece a $ K(\alpha) $.

Ahora sabemos que $ \alpha $ es algebraico si y solo sí existe un polinomio mónico irreducible en $ K $ tal que $ \alpha $ es raíz de este polinomio en $ K(\alpha) $. Sea $ x^n+a_{n-1}x^{n-1}+\cdots+a_1x+a_0 $ este polinomio. Evaluando este polinomio por $ \alpha $ obtenemos $ \alpha^n+a_{n-1}\alpha^{n-1}+\cdots+a_1\alpha+a_0 = 0$, luego $ \alpha^n+a_{n-1}\alpha^{n-1}+\cdots+a_1\alpha=-a_0 $ y entonces factorizando $ \alpha $ y multiplicando por el inverso de $ -a_0 $
obtenemos la expresión $ \alpha*(-a_0)^{-1}(\alpha^{n-1}+a_{n-1}\alpha^{n-2}+\cdots+a_1)=1 $, es decir que $ \alpha^{-1} = (-a_0)^{-1}(\alpha^{n-1}+a_{n-1}\alpha^{n-2}+\cdots+a_1)  $. Esto a su vez implica que $ K(\alpha) \in K[\alpha] $ pues logramos escribir el inverso de $ \alpha $ como un elemensto de $ K[\alpha] $.
\end{proof}
\item Muestre que $ \alpha $ es trascendente sobre $ K $ si y sólo si $ K[\alpha]  \cong K[x] $.

\begin{proof}
Tenemos que el homomorfismo $ \phi:K[x] \to K[\alpha] $ donde $ \phi $ es evaluación por $ \alpha $ es un homomorfismo sobreyectivo. Lo único que resta es demostrar que este homomorfismo es inyectivo si y solo sí $ \alpha $ es trascendente.

De hecho sabemos que $ \alpha $ es trascendente si y solo sí $ \alpha $ no es algebraico, es decir, si y solo si no existe un polinomio $ P(x) $ distinto al polinomio 0 tal que $ P(x)=0 $. Pero esto es equivalente a que el kernel de $ \phi $ es igual a $ \{0\} $, lo que significa que el homomorfismo es además inyectivo y por lo tanto es un isomorfismo. 
\end{proof}

\end{enumerate}

\item \begin{enumerate}
\item Sean $ M/L $ y $ L/K $ extensiones de cuerpos, y suponga que ambas extensiones son algebraicas. Muestre
que $ M/K $ es una extensión algebraica.

\begin{proof}

Esto es equivalente a demostrar que $ [K(\alpha):K]< \infty $ para cualquier $ \alpha \in M $.

Primero como $ M/L $ es algebráico tenemos que $ [L(\alpha):L]< \infty $ para todo $ \alpha \in M $.

Es decir que existe algún polinomio $ x^n+a_{n-1}x^{n-1}+ \cdots a_1x+a_0 $, con coeficientes en $ L $ tal que $ \alpha $ es raíz de este polinomio.

Entonces si consideramos la extensión $ K(a_0,a_1,\cdots, a_{n-1}) $ esta es una expansión finita porque es finitamente generada por elementos que son algebraicos en $ K $. El polinomio minimal de $ \alpha $ pertenece a esta extensión por lo que $ [K(a_0,a_1,\cdots a_{n-1})(\alpha):K(a_0,a_1,\cdots a_{n-1})] < \infty $. 
Finalmente, por lema de las torres tenemos que $ [K(a_0,a_1,\cdots a_{n-1})(\alpha):K]=[K(a_0,a_1,\cdots a_{n-1})(\alpha):K(a_0,a_1,\cdots a_{n-1})][K(a_0,a_1,\cdots a_{n-1}):K] < \infty $.
Pero también tenemos que $ [K(a_0,a_1,\cdots a_{n-1})(\alpha):K]=[K(a_0,a_1,\cdots a_{n-1})(\alpha):K(\alpha)][K(\alpha):K] < \infty $ lo que implica que $ [K(\alpha):K] < \infty $, es decir, que $ \alpha $ es algebraico en $ K $. 
\end{proof}

 \item En clase vimos que si $ L/K $ es una extensión finita entonces la extensión es algebraica. Pruébelo de nuevo acá.
 \begin{proof}
 Tome cualquier $ \alpha \in L $. Por hipotesis tenemos que $ [L:K]<\infty $. Además tenemos que $ K(\alpha) \subseteq L $.
 Luego por el lema de torres tenemos que $ [L:K]=[L:K(\alpha)][K(\alpha):K] < \infty $ lo que implica que $ [K(\alpha):K] < \infty $, es decir, que $ \alpha $ es algebraico en $ K $. 
 \end{proof}
 \item Muestre mediante el siguiente ejemplo que el converso del anterior no es cierto. Sea $ S $ el sub-conjunto
 de los números reales dado por la raíces primas de 2 i.e.,
 \begin{equation}
 S := \{2^{1/p} : p \text{ es primo}\}.\nonumber
 \end{equation} 

Muestre que $ \mathbb{Q}(S)=\mathbb{Q} $ es una extensión algebraica tal que $ [\mathbb{Q}(S) : \mathbb{Q}] = \infty $.

\begin{proof}
Primero para demostrar que $ \mathbb{Q}(S) $ es algebraico solo basta demostrar que los elementos en $ S $ son algebraicos, porque los demás elementos en $ \mathbb{Q}(S) $ se pueden ver como sumas, multiplicaciones o divisiones finitas de elementos de $ \mathbb{Q} $ y $ S $ que por lo tanto también son algebraicos. Y cualquier elemento $ 2^{1/p}\in S $ es algebraico en $ \mathbb{Q} $ porque el polinomio $ x^p-2 $ tiene como raíz a $ 2^{1/p} $. 

Además, por el criterio de Einsenstein $ x^p-2 $ es irreducible en $ \mathbb{Q} $ para cualquier $ p $ primo en $ \mathbb{Z} $ por lo que este polinomio es el minimal. Ahora para demostrar que $ [\mathbb{Q}(S):\mathbb{Q}] = \infty $ suponga por contradicción que $ [\mathbb{Q}(S):\mathbb{Q}] = n < \infty $. Entonces todo elemento en $ \mathbb{Q}(S) $ deberia ser raiz de algún polinomio irreducible $ \mathbb{Q} $ de grado menor o igual a $ \mathbb{n} $, pero sabemos que el conjunto de primos es infinito. Luego, podemos encontrar algún primo $ p > n $ y entonces el elemento $ 2^{1/p} \in S $ cuyo polinomio minimal es $ x^p-2 $ y por lo tanto no hay ningún polinomio de grado menor o igual a $ n $ para el que $ 2^{1/p} $ es raíz. Por lo tanto llegamos a una contradicción.  
\end{proof}
\end{enumerate}
\item Para esta pregunta asuma que $ e:=\sum_{n=0}^{\infty}\frac{1}{n!} $ y $ \pi : = \int_{0}^\infty \frac{2}{1+x^2}dx \in \mathbb{R} $ son trascendentes sobre $ \mathbb{Q} $. Sean $ \alpha=e+\pi $ y $ \beta = e\pi $. Muestre que al menos uno entre $ \alpha $ y $ \beta $ no es algebráico sobre $ \mathbb{Q} $. ( Nota: En principio
uno de ellos puede ser algebraico, pero es un problema abierto decidir si los dos son trascendentes, de
hecho no sé sabe si quiera si son irracionales.)

\begin{proof}
Considere el polinomio $ x^2-\alpha x+\beta $. Efectivamente este polinomi se puede descomponer en $ \mathbb{R} $ como $ (x-\pi)(x-e) $. La fórmula cuadrática nos da una expresión para calcular las raices de este polinomio.

\begin{equation}
x_{1,2}=\frac{\alpha \pm \sqrt{\alpha^2-4\beta}}{2} \nonumber
\end{equation}

Donde $ x_1 $ y $ x_2 $ van a ser $ \pi $ o $ e $. Sabemos que la suma, resta y división de algebraicos es algebraico. Tan solo falta demostrar que la raiz de algebraicos es algebraica y esto es así. Suponiendo que la raiz no esta contenida en una extension algebraica podemos hacer una extensión de grado 2 sobre esta extensión para agregarla (partiendo el anillo de polinomios por el polinomio $ x^2-\alpha $, por ejemplo) y la extensión vista desde $ \mathbb{Q} $ seguiria siendo algebraica por el punto 3(a) de esta tarea.

Pero esto implicaria que tanto $ e $ como $ \pi $ son algebráicos lo cual es una contradicción. Concluimos que por lo menos uno de los dos entre $ \alpha $ y $ \beta $ son trascendentes.
\end{proof}

\item Sean $ p_1 $ y $ p_2 $ primos distintos. Suponga que $ m_i $, para $ i = 1, 2 $, son enteros positivos tales que $ m_i $ no es
una potencia $ p_i $-ésima perfecta. Sean $ \mu_i $ los reales positivos definidos por las dos ecuaciones $ \mu_i^{p_i} = m_i
(i = 1; 2) $.
\begin{enumerate}
\item Fije $ i \in \{1,2\} $ y sea $ F $ un subcuerpo de $ \mathbb{R} $ que no contiene a $ \mu_i $. Si $ \mu_i^n \in F $
para algún entero no
negativo $ n $, muestre que $ p_i | n $.
\begin{proof}
Vamos a demostrar primero que para cualquier $ \alpha \in F $ se tiene que si $ \alpha^n \in F $ y $ \alpha^m \in F $ entonces $ \alpha^{g.c.d(n,m)} \in F $. La demostración es utilizando la identidad de Bezout, que nos dice que existen $ a,b \in \mathbb{Z} $ tales que $ am+bn=g.c.d(m,n) $. Note que si $ a= -c $ es negativo entonces se toma $ \alpha^a=(\alpha^{-1})^c $ Entonces tenemos que $ (\alpha^m)^a(\alpha^n)^b = \alpha^{an+bm}=(\alpha^{g.c.d(m,n)} \in F $.

Ahora la otra cosa que cabe notar es que por nuestra suposición $ \mu_i^{p_i}=m_i \in \mathbb{Z} \subseteq \mathbb{Q} $. En una tarea anterior demostramos que todo campo de caracteristica 0, como $ \mathbb{R} $ contiene una copia de $ \mathbb{Q} $. Entonces cualquier subcampo $ F $ contiene a $ \mathbb{Q} $ por lo que concluimos que $ \mu_i^{p_i}=m_i \in F $. 

Entonces si suponemos por contradicción que existe $ n $ tal que $ p_i  \not|\; n $ y $ \mu_i^n \in F $ tendriamos primero que $ g.c.d(n,p_i)=1 $ y por lo discutido anteriormente concluimos que $ \alpha^1=\alpha \in F $ lo cual es una contradicción. 
\end{proof}
\item De nuevo, si $ \mathbb{R}/F $ es tal que $ \mu_i \not \in F $ muestre que $ [F(\mu_i) : F] = p_i $. Deduzca de lo anterior que el
polinomio $ x^{p_i} - m_i \in F[x] $ es irreducible. (Sugerencia para la primera parte: Considere el término
constante del polinomio minimal de $ \mu_i $ sobre $ F $.)

\begin{proof}
Sabemos que $ \mu_i $ es una raiz del polinomio $ x^{p_i}-m_i $. Luego el polinomio minimal de $ \mu_i $ divide a este polinomio. Si nos extendemos al campo algebraicamente cerrado $ \mathbb{C} $ podemos factorizar el polinomio minimal como $ (x-\alpha_1)(x-\alpha_2)\cdots (x-\alpha_m) $ donde $ m $ es el grado de este polinomio que debe ser menor o igual a $ p_i $.

Ahora las otras raices de $ x^{p_i}-m_i $ estan dadas por $ \mu_i \omega_j $ donde $ \omega_j $ es una $ p_i $-ésima raiz de la unidad.

Ahora si consideramos el término constante del polinomio minimal este debe ser igual a $ \alpha_1\cdots \alpha_m = \mu^m(\omega_1\cdots \omega_m) \in F $. Pero recordemos que $ F $ es un subcampo de los reales, luego como $ \mu \in \mathbb{R} $ concluimos que $ (\omega_1\cdots \omega_m) \in \mathbb{R}$. Además, se puede observar que $ (\omega_1\cdots \omega_m)^p_i = 1\cdots 1 = 1 $, pero las únicas raices de la unidad reales son $ 1 $ o $ -1 $. Por lo tanto, el término constante es $ \mu^m $ o $ -\mu^m $. En cualquier caso, por el punto anterior tenemos que $ p_i|m $ y por lo tanto $ m $ debe ser igual a $ p_i $. Como este es el grado del polinomio minimal concluimos finalmente que $ [F(\alpha):F]=p_i $. 
\end{proof}
\item  Sea $ K $ el sub-cuerpo de $ R $ dado por $ K = \mathbb{Q}[\mu_1 + \mu_2] $. Muestre que $ [K : \mathbb{Q}] = p_1p_2 $
\begin{proof}
Tenemos que $ \mu_1 $ y $ \mu_2 $ son algebraicos, luego su suma es algebraica y por el punto 2 tenemos que $ \mathbb{Q}[\mu_1 + \mu_2] = \mathbb{Q}(\mu_1 + \mu_2) $. Ahora por lo demostrado anteriormente tenemos que ni $ \mu_1 $ ni $ \mu_2 $ pertenecen a $ \mathbb{Q} $. Luego $ [K(\mu_1):K]=p_1 $ y además $ \mu_2 \not \in \mathbb{Q} $ puesto que si estuviera entonces $ p_2 $ deberia dividir a $ p_1 $ lo cual es una contradicción. Luego tenemos que $ [K(\mu_1,\mu_2):K(\mu_1)]=p_2 $ y por el lema de las torres tenemos que $ [K(\mu_1,\mu_2):K]=p_1p_2 $.

Ahora también tenemos que $ K(\mu_1,\mu_1+\mu_2) = K(\mu_2,\mu_1+\mu_2) = K(\mu_1,\mu_2) $. Valiendonos de nuevo del lema de torres tenemos que $ p_1p_2=[K(\mu_1,\mu_1+\mu_2):K]=[K(\mu_1,\mu_1+\mu_2):K(\mu_1+\mu_2)][K(\mu_1+\mu_2:K] $ y $ p_1p_2 = [K(\mu_2,\mu_1+\mu_2):K]=[K(\mu_1,\mu_1+\mu_2):K(\mu_1+\mu_2)][K(\mu_1+\mu_2:K] $. Ahora $ [K(\mu_1,\mu_1+\mu_2):K(\mu_1+\mu_2)] $ puede ser 1 o $ p_2 $ dependiendo de sí $ \mu_1 $ esta contenido o no y de igual manera con $ [K(\mu_2,\mu_1+\mu_2):K(\mu_1+\mu_2)] $. Si alguno de estos es 1 entonces concluiriamos por el lema de las torres que  $ [K(\mu_1+\mu_2):K]=p_1p_2 $, pero si suponemos que ninguno de los 2 es 1 llegariamos rapidamente a una contradicción porque el lema de las torres nos permitiria concluir que $ [K(\mu_1+\mu_2):K] = p_1 $ y $ [K(\mu_1+\mu_2):K] = p_2 $ al mismo tiempo. Luego $ [K(\mu_1+\mu_2):K] = p_1p_2 $
\end{proof}
\end{enumerate}

\item Sea $ L/K $ una extensión de cuerpos. Un subconjunto $ T \subseteq L $ se dice algebraicamente independiente sobre
$ K $ si para todos $ t_1,\cdots , t_n \in T $ se tiene que si $ p(x_1,\cdots, x_n) \in K[x_1, \cdots, x_n] $ es tal que $ p(t_1, \cdots , t_n) = 0 $ entonces $ p(x_1,\cdots, x_n) $ es el polinomio 0.

\begin{enumerate}
\item Muestre que si $ T = \{t\} $ consiste de un solo elemento entonces $ T $ es algebraicamente independiente
sobre $ K $ si y sólo si $ t $ es trascendente sobre $ K $.
\begin{proof}
Por nuestra suposición que solamente hay un elemento evaluar sobre un polinomio de $ x_n $ variables es equivalente a evaluar sobre un polinomio de una sola variable que se obtiene sustituyendo cada una de las variables $ x_i $ por el indetermindado $ x $. Efectivamente tenemos que $ t $ es trascendente si y solo si no existe un polinomio con coeficientes en $ K $ tal que $ t $ sea raíz de este polinomio Por lo anterior, concluimos que $ T $ es algebraicamente independiente.  
\end{proof}
\item  Muestre que toda extensión $ L/K $ contiene un subconjunto algebraicamente independiente maximal $ T $.

\begin{proof}
Vamos a utilizar el Lema de Zorn sobre la colección de subconjuntos alg ind de $ L $. Esta colección es no vacía si tomamos en cuenta el vacío, que por vacuidad cumple todas las condiciones de subconjuntos alg ind. Ahora sea $ C $ una cadena de colecciones alg ind. Vamos a demostrar que $ B = \bigcup_{A \in C} A $ es una cota para la cadena $ C $. Claramente $ A \subseteq B $ para cualquier $ A \in C $. Entonces solo resta demostrar que $ B $ es un conjunto alg ind. Entonces tome cualesquiera $ t_1,\cdots, t_n \in T $. Por cada $ t_i $ existe un $ A_i $ en la cadena y como son finitos yo puedo tomar algun $ A_n $ tal que los contenga a todos. Luego como $ A_n $ es un conjunto alg ind concluimos que no hay ningun polinomio distinto de 0 que al evaluarlo con estos parametros de 0. Por lo tanto, concluimos que $ B $ es una cota y luego podemos utilizar el Lema de Zorn para concluir que debe existir algún subconjunto alg ind máximal.
\end{proof}

\item  Sea $ T $ un conjunto alg ind maximal para la extensión $ L/K $. Muestre que la extensión $ L/K(T) $ es
algebráica.

\begin{proof}
Si suponemos por contradicción que $ L/K(T) $ no es algebraico entonces existiría al menos un elemento $ \alpha \in L $ trascendente en $ L/K(T) $. Claramente $ \alpha \not \in T $ porque cualquier elemento en $ T $ es algebraico en $ L/K(T) $. Entonces si tomamos el conjunto $ T \cup \{\alpha\} $ este es alg ind. Si esto no fuera así es porque existe algún polinomio sobre $ K(T) $ tal que $ \alpha $ es raíz, pero entonces no sería trascendente. Llegamos pues a una contradicción con la maximalidad de $ T $.    
\end{proof}

\end{enumerate}
Se puede mostrar que dada una extensión $ L/K $ cuales quiera dos conjuntos alg ind maximales (conocidos
como bases de trascendencia) tienen el mismo cardinal. A éste cardinal se le conoce como el grado de
trascendencia de $ L $ sobre $ K $ y se denota por trdeg $ (L/K) $.

\end{enumerate}

\end{document}